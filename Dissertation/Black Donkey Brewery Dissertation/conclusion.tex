\chapter{Conclusion}
This section discusses the final conclusions and outcomes of the project in detail. It also looks back at how the project compared to the original outcomes set in the project in comparison with the final product produced.

\section{Conclusion Overview}
The goal of this project was to create a system that organised and handled all the revenue processes that a brewing company must carry out. Its aim was to reduce the current system of loads of excel and word documents that required manual calculations to be carried out and manual storage, all into an online automated system which stored this data, also allowing users to edit and create more data that would have the necessary calculations carried out each time. The user should also be able to view or delete any selection of data they wish. This data also needs to be printable. The client wanted this system to significantly speed up the process that was required each month. They felt that this application was a massive requirement in their industry as this is a common problem across similar companies.

\section{Context and Objectives Review}
The objectives set at the beginning of the project were met to a high standard. A great understanding of the requirements were obtained due to great lines of communication between all parties involved. The research conducted into the methodologies and technologies to use was greatly beneficial as this allowed organization of all aspects of the project and ensured all areas were sufficiently studied and researched. The system architecture was well designed as all aspects from the beginning were well thought out and cohesion between all designed parts worked well throughout all of development. The system produced was of a sufficient standard in order to complete all the aforementioned requirements by the client.\par
The system meets all the requirements specified as it allows users to create, update, delete and view information for brews made, inventories created and stock returns produced and ability to view a warrant. These all rely on previously created data to produce their own data, other than the exception of a brew. The data has a mass amount of calculations carried out on the server side. This closely interlinked data results in a complex system which is carefully tested to ensure correctness. The data is all stored in a database and the full application is on a cloud system.\par
The project scope was sufficient for a year long project. This was largely due to the complexity of the backend areas like the calculations which take a long period of time to produce and ensure correctness. The testing process also was critical in this area as data results are going to be used for government documentation on the business. Using all new technologies with the exception of the physical database (method of connecting - PyMongo - was never used before) also led to the project being worthy of the module as this all took time to learn these new technologies to a level of competency in which a project of this magnitude could be completed. Working with a client also provided great challenges for this module that would not be experienced without. Interactions outside of college produced a great understanding of working at an industry level. Full stack development is a great method of working for this module due to the variety of different technologies and problems one encounters and must overcome to produce software of a good standard.\par
The risks involved in the project were all monitored throughout the project lifecycle to ensure that they did not come into effect during development. Planning out the risks before development meant that the possibility of accidental introduction of these risks was greatly reduced. An example is security of data. This was integral to the project and authentication is needed at every level to access any form of data.\par
As a result of reaching all levels of requirements and objectives set at the beginning of the project and the client's satisfaction with the final product, the project can be deemed a massive success.

\section{System evaluation results}
Testing of the system was carried out to a large degree. This was very important due to the systems use in revenue calculations of a company. Testing was a major part of the development for the system and was carried out in a variety of ways to ensure all areas of the system achieved what was required of it. During the testing process it was clear due to the many bugs caught that this is a crucial part of any project and is evident to see why companies have teams who are solely focused on testing. Areas such as technology limitations were also encountered and highlight the flaws in free architecture in comparison with paid infrastructure. 

\section{Project outcomes}
This section details the final outcomes from the project, they are as follows:
\begin{itemize}
    \item A web application which carries out all of the requirements specified for it.
    \item An API server which sends and receives data as well as carrying out any necessary data manipulation in the form of calculations on the data received.
    \item Database stores all user data and information and it is possible to query data whenever needed from the database.
    \item Both the web application and server are hosted on cloud servers and communicate between each other.
\end{itemize}

\section{Future Adaptations}
Although this system is created solely for one client. It is easily adaptable to be expanded to more users. The client wanted their own brewery application with the view of possibly marketing it to other breweries with similar issues. This restriction resulted in making the system only allow authorized users to access data. Rather than a sign up and login system that could allow all breweries to create their own accounts and access their data. This could be a possible change made in the future if the client wishes to make the website open to all breweries. \par
A possible server change could be instantiated if the project were to be repeated again. Due to using a free tier hosting service, both servers take a period of time to start before being able to access them. This is a limitation on the project and with paid hosting servers, a better and substantially faster service could be provided for users.\par
Another change that could be made is aesthetically. The website could look much better in the hands of a proper web page graphic designer, though stressed by the client this was not an issue from the beginning. The prospect of a better looking web application is one that could be achieved. This would definitely be a priority were the application to be made again, but the main aim was to get a working system up and running.\par
More changes in terms of features could be added if not constrained to the requirements of the client, for example an API that links all the breweries on google maps using geolocation to place markers at each of the brewery locations and allows for orders to take place between them. However, the client rejected any idea of enhancements.

\section{Final Conclusion}
The project was a great learning experience. The interaction of working with a client is one which will be of great benefit for working in industry. The usage of new technologies also greatly improved researching abilities as well as coding capabilities. The project highlights the benefit of software development methodologies and their usage within industry. Time management also played a large role and shows how important deadlines are within companies. This project displays the benefits of setting up a project with the ability to change around new requirements. This project provided great experience and knowledge for working within industry projects in the future.